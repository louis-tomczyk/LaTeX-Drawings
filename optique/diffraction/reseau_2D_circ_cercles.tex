%%%%%%%%%%%%%%%%%%%%%%%%%%%%%%%%%%%%%%%%%%%%%%%%%%%%%%%%%%%%%%%%%%%%%%%%%%%%%%%%%%%%%%
% author                : louis tomczyk
% date of production    : 2024-12-05
% licence               : cc-by-nc-sa
%                         Attribution - Non-Commercial - Share Alike 4.0 International
%%%%%%%%%%%%%%%%%%%%%%%%%%%%%%%%%%%%%%%%%%%%%%%%%%%%%%%%%%%%%%%%%%%%%%%%%%%%%%%%%%%%%%

\documentclass[tikz,border=5mm]{standalone}
\usepackage{pgfmath}
\usepackage{comment}
\begin{document}

\begin{tikzpicture}[scale=1]
    \pgfmathsetmacro{\minRadius}{0.2}     % Rayon du cercle le plus petit
    \pgfmathsetmacro{\maxRadius}{1.0}     % Rayon du cercle le plus grand
    \pgfmathsetmacro{\nCircles}{5}        % Nombre de cercles concentriques
    \pgfmathsetmacro{\angleStep}{30}      % Espacement angulaire (en degrés)
    \pgfmathsetmacro{\pointRadius}{0.02} % Rayon des points

    % Boucle pour les cercles concentriques
    \foreach \circleIndex in {1,...,\nCircles} {
        \pgfmathsetmacro{\currentRadius}{\minRadius + (\circleIndex-1) * (\maxRadius - \minRadius) / (\nCircles-1)}
        % Boucle pour les points sur le cercle actuel
        \foreach \angle in {0,\angleStep,...,359} {
            % Calcul des coordonnées polaires
            \pgfmathsetmacro{\x}{\currentRadius * cos(\angle)}
            \pgfmathsetmacro{\y}{\currentRadius * sin(\angle)}
            % Dessiner le point
            \fill[black] (\x, \y) circle (\pointRadius);
        }
    }



\end{tikzpicture}

\end{document}
